\chapter[Appendix]{}
\label{app:spin-predictions}

\section{Singlet state representation on different bases}
We prove here that the state of equation (\ref{eq:singlet-state}) takes the same form on all orthonormal bases of the $S_x$, $S_y$ and $S_z$ observables.

Let us prove that the state:
\begin{equation}
  |\chi\rangle_{\mathbf{\hat{z}}} = \frac{1}{\sqrt{2}} \left( |\mathbf{\hat{z}}, +\rangle_1 |\mathbf{\hat{z}}, -\rangle_2 - |\mathbf{\hat{z}}, -\rangle_1 |\mathbf{\hat{z}}, +\rangle_2 \right)
  \label{eq-singlet-state-z}
\end{equation}
has the same form on the tensor basis obtained from the two bases $\left( |\mathbf{\hat{x}}, +\rangle_i, |\mathbf{\hat{x}}, -\rangle_i \right), i = 1, 2$ the same argument can be adapted to the $\mathbf{\hat{y}}$ case.

Let us write the $S_x$ operator as a $2 \times 2$ matrix on the $\left( |\mathbf{\hat{z}}, +\rangle_i, |\mathbf{\hat{z}}, -\rangle_i \right)$ basis:
\begin{equation*}
  S_x =
  \begin{pmatrix}
    0 & 1\\
    1 & 0
  \end{pmatrix}
\end{equation*}
in units $\frac{\hbar}{2} = 1$. This operator can be diagonalized leading to the eigenvectors:
\begin{equation}
  \begin{split}
    |\mathbf{\hat{x}}, +\rangle_i &= \frac{1}{\sqrt{2}} \left( |\mathbf{\hat{z}}, +\rangle_i + |\mathbf{\hat{z}}, -\rangle_i \right),\\
    |\mathbf{\hat{x}}, -\rangle_i &= \frac{1}{\sqrt{2}} \left( |\mathbf{\hat{z}}, +\rangle_i - |\mathbf{\hat{z}}, -\rangle_i \right)
  \end{split}
  \label{eq:sx-eingenvectors}
\end{equation}
with eigenvalues $+ 1$ and $- 1$ respectively. Inverting the two equations (\ref{eq:sx-eingenvectors}) and substituting into (\ref{eq-singlet-state-z}) leads to the desired result.

\section{Quantum mechanical predictions on GHZ states}
%\subsection{Perfect correlation in systems of three spin-1/2 particles}
%Consider the state in (\ref{eq:ghz-state}), i.e.:
%\begin{equation*}
%  |\chi\rangle = |+\rangle_1 |+\rangle_2 |+\rangle_3 - |-\rangle_1 |-\rangle_2 |-\rangle_3,
%\end{equation*}
%we will show here that the result of a measurement of either the $S_x$ or $S_y$ spin component on any of the three particles can be predicted with certainity by appropriate measurements on the other two.
%
%Consider for example a $S_y$ component measurement on particle 3, analogous arguments can be applied to the other combinations of particles and spin components.
%
%We start by measuring $S_x$ of particle 1, 
%
%\subsection{Measurements of the product of spin components on GHZ states}
%The order in which we prove the results in this section is inverted with respect to the order in which they are used in Chapters \ref{chap:ghz-theorem}. This is because the first result used can be easily proved as a corollary of the second.

\subsection{Perfect correlation in systems of three spin-1/2 particles}
Consider three particles in the state:
\begin{equation}
  |\chi\rangle = |+\rangle_1 |+\rangle_2 |+\rangle_3 - |-\rangle_1 |-\rangle_2 |-\rangle_3,
  \label{eq:ghz-state-app}
\end{equation}
we will show here that the result of a measurement of either the $S_x$ or $S_y$ spin component on any of the three particles can be predicted with certainty by appropriate measurements on the other two.

Let us take, for example, a $S_y$ spin component measurement on particle 3, analogous arguments can be applied to the other combinations of particles and spin components.

To predict the result of this measurement we consider the first of the triples in (\ref{eq:xyy-observables-triplets}), that is:
\begin{equation*}
  \left( S_{1x} \times \mathbb{1}_2 \times \mathbb{1}_3,~~~ \mathbb{1}_1 \times S_{2y} \times \mathbb{1}_3,~~~ \mathbb{1}_1 \times \mathbb{1}_2 \times S_{3y} \right).
\end{equation*}
We will show that after the first two observables in the triple are measured the third particle will be left in an eigenstate of the $\mathbb{1}_1 \times \mathbb{1}_2 \times S_{3y}$ observable.

Writing the state (\ref{eq:ghz-state-app}) on a basis of eigenstates of the three observables in the triplet considered leads to:
\begin{equation*}
  \begin{split}
    |\chi\rangle = \frac{1}{2} \left( i |\mathbf{\hat{x}}, +\rangle_1 |\mathbf{\hat{y}}, +\rangle_2 |\mathbf{\hat{y}}, -\rangle_3 + i |\mathbf{\hat{x}}, +\rangle_1 |\mathbf{\hat{y}}, -\rangle_2 |\mathbf{\hat{y}}, +\rangle_3 +\right.\\
    \left. + |\mathbf{\hat{x}}, -\rangle_1 |\mathbf{\hat{y}}, +\rangle_2 |\mathbf{\hat{y}}, +\rangle_3 - |\mathbf{\hat{x}}, -\rangle_1 |\mathbf{\hat{y}}, -\rangle_2 |\mathbf{\hat{y}}, -\rangle_3 \right).
  \end{split}
\end{equation*}
Thus to any of the four possible combinations of results for the first two measurements there corresponds only one possible result for the third observable.

\subsection{Measurements of the product of spin components on GHZ states}
We will show here that the state (\ref{eq:ghz-state}), i.e.:
\begin{equation*}
  |\chi\rangle = |+\rangle_1 |+\rangle_2 |+\rangle_3 - |-\rangle_1 |-\rangle_2 |-\rangle_3
\end{equation*}
is an eigenstate of all of the observables (\ref{eq:xyy-observables}), i.e.:
\begin{equation}
  \begin{split}
    S_{1x} \times S_{2y} \times S_{3y}\\
    S_{1y} \times S_{2x} \times S_{3y}\\
    S_{1y} \times S_{2y} \times S_{3x}
  \end{split}
  \label{eq:xyy-operators-app}
\end{equation}
with eigenvalue $+ 1$ and it is also eigenstate of the observable (\ref{eq:xxx-observable}), i.e.:
\begin{equation*}
  S_{x1} \times S_{2x} \times S_{3x}
\end{equation*}
this time with eigenvalue $- 1$.

Let us prove this statement for the first observable in (\ref{eq:xyy-operators-app}) the same argument can be adapted to the other observables. We begin with the spin of particle 1:
\begin{equation*}
  \begin{split}
    S_{1-} |\chi\rangle &= 2 |-\rangle_1 |+\rangle_2 |+\rangle_3\\
    S_{1+} |\chi\rangle &= - 2 |+\rangle_1 |-\rangle_2 |-\rangle_3
  \end{split}
\end{equation*}
that gives:
\begin{equation*}
  |\chi\rangle' := S_{1x} |\chi\rangle = |-\rangle_1 |+\rangle_2 |+\rangle_3 - |+\rangle_1 |-\rangle_2 |-\rangle_3.
\end{equation*}
For particle 2 we have:
\begin{equation*}
  \begin{split}
    S_{2-} |\chi\rangle' &= 2 |-\rangle_1 |-\rangle_2 |+\rangle_3\\
    S_{2+} |\chi\rangle' &= - 2 |+\rangle_1 |+\rangle_2 |-\rangle_3
  \end{split}
\end{equation*}
thus:
\begin{equation*}
  |\chi\rangle'' := S_{2y} |\chi\rangle' = - \frac{1}{i} \left( |+\rangle_1 |+\rangle_2 |-\rangle_3 + |-\rangle_1 |-\rangle_2 |+\rangle_3 \right).
\end{equation*}
The same goes through for particle 3 and leads to:
\begin{equation*}
  |\chi\rangle''' := S_{3y} |\chi\rangle'' = |+\rangle_1 |+\rangle_2 |+\rangle_3 - |-\rangle_1 |-\rangle_2 |-\rangle_3 = |\chi\rangle.
\end{equation*}
which is the desired result.
